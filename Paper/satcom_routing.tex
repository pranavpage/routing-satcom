\documentclass[10pt,conference, draft, letterpaper]{IEEEtran}
\usepackage{cite}
\usepackage{amsmath,amssymb,amsfonts}
\usepackage{algorithmic}
\usepackage{graphicx}
\usepackage{textcomp}
\usepackage{xcolor}
\newcommand{\rough}[1]{{\color{red} #1}}
\def\BibTeX{{\rm B\kern-.05em{\sc i\kern-.025em b}\kern-.08em
		T\kern-.1667em\lower.7ex\hbox{E}\kern-.125emX}}
\title{Distributed Congestion Control in LEO Satellite Networks}
%\author{\IEEEauthorblockN{ Pranav Page}
%	\IEEEauthorblockA{\textit{Dept. of Electrical Engineering} \\
%		\textit{IIT Bombay}\\
%		Mumbai, India}
%	\and 
%	\IEEEauthorblockN{ Kaustubh Bhargao}
%	\IEEEauthorblockA{\textit{Dept. of Electrical Engineering} \\
%		\textit{IIT Bombay}\\
%		Mumbai, India}
%	\and 
%	\IEEEauthorblockN{ Hrishikesh Baviskar}
%	\IEEEauthorblockA{\textit{Dept. of Electrical Engineering} \\
%		\textit{IIT Bombay}\\
%		Mumbai, India
%}}

\begin{document}
\maketitle	
	\begin{abstract}
Satellite communication in LEO constellations has become an emerging topic of interest. Due to the high number of LEO satellites in a typical constellation, a centralized algorithm for minimum-delay packet routing would incur significant signaling and computational overhead. We can exploit the deterministic topology of the satellite constellation to calculate the minimum-delay path between any two nodes in the satellite network, but that does not take into account the traffic information at the nodes along this minimum-delay path.\\
We propose a distributed probabilistic congestion control scheme to minimise end-to-end delay. In the scheme, each satellite, while sending a packet to its neighbour, adds a header with a simple metric indicating its own congestion level. The decision to route packets is taken based on the latest traffic information received from the neighbours. We build this algorithm onto the Datagram Routing Algorithm, which provides the minimum delay path, and the decision for the next hop is taken by the congestion control algorithm. We compare the proposed congestion control mechanism with the existing congestion control used by the DRA, and show improvements over the same.
	\end{abstract}
\section{Introduction}
\rough{What is the problem?}\\
With the advent of cost-effective space launch systems, the feasibility of space-based communication networks has turned into a reality. Dense Low Earth Orbit constellations such as Starlink and OneWeb have joined sparse constellations like Iridium in orbit and are operational. Communication using LEO satellite constellations is favoured over GEO satellites due to the much lower ground-to-satellite propagation delay. \\
The challenges faced by satellite constellations are very different from those encountered by terrestrial networks. The nodes in a satellite constellation are constantly moving relative to the ground, so association and handover in a ground-to-satellite link are non-trivial problems. The satellites typically deployed in a constellation are small in size (about 150 kg), which results in limited on-board processing and storage capacity. In addition, the small size of the satellites leads to difficulties in antenna pointing. The inter-satellite links are also characterized by high propagation and transmission delays and high BER. Limited on-board storage capacity leads to packet drops when the nodes get congested, thus degrading the flow of packets. The network has two types of inter-satellite links (ISLs), namely \textit{intra-plane} ISLs, which are the ISLs between two neighbouring satellites in the same orbital plane and \textit{inter-plane} ISls, which are the ISLs between two neighbouring satellites in different orbital planes. The inter-plane ISLs are difficult to maintain in the polar regions due to the rapid movement of satellites and switching of relative positions.\\
Due to the dynamic nature of the satellite constellation, paths computed at a central location and sent to the nodes in the network would need a lot of transmissions and computations involving a dense network. Thus, a distributed routing and congestion control algorithm is preferred. The DRA takes advantage of the spherical geometry of the network and calculates the optimum minimum delay path using the relative positions of the nodes. After that, it is the job of the congestion control algorithm to pick the next hop for a packet to reach its destination with the minimum queueing and propagation delay. The problem of choosing the next hop for the packet in the presence of congestion is the problem that this work focuses on. The choice has to be made locally, and without knowledge of the congestion level of every node along the minimum delay path.\\
\rough{Why is it interesting and important?}\\
A distributed congestion control algorithm that can deal with uneven node congestion levels to route packets from source to destination with low packet drops and end-to-end delay would be easy to implement on-board, and would offer better QoS. \\
\rough{Why is it hard?}\\
The problem of choosing an optimum schedule which minimizes the total end-to-end delay for a given set of packets has been shown to be NP-hard in \cite{opt_schedule}. Thus, heuristics-based approaches have been used to tackle this problem. In particular, \cite{ekici-datagram}\cite{ekici-dist} use a basic threshold on the outgoing buffer to determine whether the link is congested. The DRA does not use local congestion information to reroute packets. We address this problem by using the packet headers as a way of conveying traffic information in the form of a single metric indicating the congestion level, and then probabilistically choosing the next hop for a packet. We simulate a typical LEO satellite constellation, and compare the performance of the DRA and our own algorithm in terms of end-to-end delay and packet drops. \rough{Add figures of improvement. Add breakdown of the paper(?)}
\section{Related works}
\section{System Model and Problem Formulation}
\section{Algorithms}
\section{Simulation setup}
\section{Results}
\section{Conclusions and Future Work}
\bibliography{refs.bib}
\bibliographystyle{ieeetr}
\end{document}